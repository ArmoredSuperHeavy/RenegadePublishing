% by default, memoir doesn't give you many options for styling title pages for book and part subdivisions -
% this overrides some of the bits to give you an easier interface into styling these pages.

% Because we're overwriting \bookpagemark and \partmark we need new names to store the book and part titles
% So they can be used 
\newcommand{\thebookname}{}
\newcommand{\thepartname}{}

% new environments to keep font changes contained
\newenvironment{bookpageenv}{}{}
\newenvironment{partenv}{}{}


% Default page stylings that are the same as the non-overriden ones
% Override these to do your restyling. Look at the memoir manual 6.4
% for info on what to override to change fonts and whatnot.
\newcommand{\bookpagestyle}[1]{
    \centering
    \normalfont
      \printbookname \booknamenum \printbooknum
      \midbookskip
    \printbooktitle{#1}\par
}

\newcommand{\partpagestyle}[1] {
    \centering
    \normalfont
    \printpartname \partnamenum \printpartnum \thepart
    \midpartskip

\printparttitle{#1}\par}

% Where the magic happens! Note that #2 in both of these contains the default styling
% if you want to use that for some reason.
\renewcommand{\bookpagemark}[2]{% 
    % Reset part and chapter counters manually (by default, their numbering is consecutive regardless of book)
    \setcounter{part}{0}
    \setcounter{chapter}{0}

    \renewcommand{\thebookname}{#1}
    \begin{bookpageenv}
        \bookpagestyle{#1}
    \end{bookpageenv}
}

\renewcommand{\partmark}[2]{%
    \renewcommand{\thepartname}{#1}
    \begin{partenv}
        \partpagestyle{#1}
    \end{partenv}
}